
\documentclass[12pt]{article}
%\usepackage[final]{pdfpages}

\usepackage{graphicx}
\graphicspath{{/Users/kevinhayes/Documents/teaching/images/}}

\usepackage{tikz}
\usetikzlibrary{arrows}

\newcommand{\bbr}{\Bbb{R}}
\newcommand{\zn}{\Bbb{Z}^n}

%\usepackage{epsfig}
%\usepackage{subfigure}
\usepackage{amscd}
\usepackage{amssymb}
\usepackage{amsbsy}
\usepackage{amsthm}
\usepackage{natbib}
\usepackage{amsbsy}
\usepackage{enumerate}
\usepackage{amsmath}
\usepackage{eurosym}
%\usepackage{beamerarticle}
\usepackage{txfonts}
\usepackage{fancyvrb}
\usepackage{fancyhdr}
\usepackage{natbib}
\bibliographystyle{chicago}

\usepackage{vmargin}
% left top textwidth textheight headheight
% headsep footheight footskip
\setmargins{2.0cm}{2.5cm}{16 cm}{22cm}{0.5cm}{0cm}{1cm}{1cm}
\renewcommand{\baselinestretch}{1.3}


\pagenumbering{arabic}

\begin{document}



%----------------------------------------------------------------------------------------------%
\section{Recurrence Relation}


A recurrence relation is an equation that recursively defines a sequence, once one or more initial terms are given: 

each further term of the sequence is defined as a function of the preceding terms.
The term difference equation sometimes refers to a specific type of recurrence relation. 

However, "difference equation" is frequently used to refer to any recurrence relation.



%--------------------------------------------------%

\noindent \textbf{Recurrence Relations}
\textit{CIS102 2004 Question 7 2 Marks}


\begin{itemize}
\item Consider the sequence given by \[ 1, 4, 7, 10, 13, \ldots\]
\item State a recurrence relation which expresses the nth term, $u_n$
, in terms of the$(n - 1)$th term, $u_{n-1}$, 
\item State a recurrence relation which expresses the nth term, $u_n$
, in terms of the first term $u_1$.
\end{itemize}


%--------------------------------------------------%

\noindent \textbf{Recurrence Relations}

\begin{itemize}
\item $u_1$ = 1 , $u_2$ = 4, $u_3$ = 7 etc 
\item Difference in successive terms is 3.
\item Therefore we can say 
\[ u_n = u_{n-1} + 3 \]
\end{itemize}

%--------------------------------------------------%

\noindent \textbf{Recurrence Relations}

\begin{itemize}
\item Difference between $u_2$ and $u_1$ is 3 (i.e. $1 \times 3$).
\item Difference between $u_3$ and $u_1$ is 6 (i.e. $2 \times 3$)
\item Difference between $u_4$ and $u_1$ is 9 (i.e. $3 \times 3$)
\item In general the difference between $u_n$ and $u_1$ is $(n-1)\times 3$.
\[ u_{n} = u_1 + 3 \times (n-1) \]
\[ u_{n} = 1 + (3n-3) = 3n-2\]
\item Equivalently
\[ u_{n+1} = u_1 + 3n = 3n+1\]
\end{itemize}

\begin{itemize}
\item Another sequence is defined by the recurrence relation 
\[ u_n = u_{n-1} + 2n-1 \] and
$u_1$ = 1.
\item Calculate $u_2$ , $u_3$ , $u_4$  and $u_5$ .
\item (Answers 1,4,9,16,25)
\end{itemize}

\[\begin{array}{|c|c|c|c|}
n    & u_{n-1} & 2n-1 & u_n \\ \hline
2      & 1         &  3   & u_2 = 4 \\ \hline
3      & 4         &  5   & u_3 = 9 \\ \hline
4      & 9         &  7   & u_4 = 16 \\ \hline
5      & 16        &  9   & u_5 = 25 \\ \hline
\end{array}\]



\end{document}
