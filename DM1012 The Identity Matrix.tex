\documentclass[]{article}
\usepackage{amsmath}
\usepackage{amssymb}
\begin{document}

\section*{The Identity Matrix }

The identity matrix or unit matrix of size n is the $n\times n$  square matrix with ones on the main diagonal and zeros elsewhere. It is denoted by $I_n$, or simply by I if the size is immaterial or can be trivially determined by the context. 

%(In some fields, such as quantum mechanics, the identity matrix is denoted by a boldface one, 1; otherwise it is identical to I.) Some mathematics books use U and E to represent the Identity Matrix (meaning "Unit Matrix" and the German word "Einheitsmatrix",[1] respectively), although I is considered more universal.
\[
I_{1}={\begin{bmatrix}1\end{bmatrix}},\ I_{2}={\begin{bmatrix}1&0\\0&1\end{bmatrix}},\ I_{3}={\begin{bmatrix}1&0&0\\0&1&0\\0&0&1\end{bmatrix}},\ \cdots ,\ I_{n}={\begin{bmatrix}1&0&\cdots &0\\0&1&\cdots &0\\\vdots &\vdots &\ddots &\vdots \\0&0&\cdots &1\end{bmatrix}}
\]
\newline
\noindent When A is a square matrix (i.e. $n\times n$), it is a property of matrix multiplication that
\[I_{N}A=AI_{n}=A.\,\]
\newline

\noindent When A is $m\times n$, it is a property of matrix multiplication that
\[I_{m}A=AI_{n}=A.\,\]

The Identity matrix itself is invertible, being its own inverse.
The Identity matrix is symmetrix, which is to say, it is it's own tranpose.
\end{document}
