
\subsection*{Limit of a Sequence}
An important property that a sequence can have is a \emph{limit}. Not all sequences have limits, but those which do can be very useful.

\textbf{Definition}\\
An infinite sequence $a_n$ is said to have a limit $L$ if as $n \to \infty$, $|a_n - L|\to 0$. If this is the case then we write
\begin{equation*}
  \lim_{n\to \infty} a_n = L
\end{equation*}
A sequence with a limit is said to \emph{converge} to that limit as $n$ goes to $\infty$. As $n$ becomes larger and larger, the numbers in the sequence become closer and closer to $L$, until the difference between $a_n$ and $L$ becomes indistinguishable; $|a_n-L|\to 0$.\\
\\
A sequence which has a limit is said to be \emph{convergent}\\
A sequence without a limit is said to be \emph{divergent}\\

\noindent {\bf Example:}\\
Consider the sequence $a_n=\frac{1}{n}$, which is known as the \emph{harmonic sequence}. Some terms in this sequence are given in the table below, and are plotted in figure \vref{fig:harmonic_sequence}
\begin{center}
\begin{tabular}{|c|c|c|c|c|c|c|c|c|c|}
\hline
 $a_1$ & $a_2$ & $a_3$ & $a_4$ & $a_5$ & $\cdots$ & $a_{100}$ & $\cdots$ & $a_{1000000}$ & $\cdots$  \\
\hline
1 & $\frac{1}{2}$ & $\frac{1}{3}$ & $\frac{1}{4}$ & $\frac{1}{5}$  & $\cdots$ & 0.01 & $\cdots $  & 0.000001 &  \\
\hline
\end{tabular}
\end{center}
\begin{figure}
  \centering
  \includegraphics[width=0.4\textwidth]{images/harmonic_sequence.eps}
  \caption{A plot of the terms in the harmonic sequence $\frac{1}{n}$. The terms converge towards the limit $0$. }
  \label{fig:harmonic_sequence}
\end{figure}
As $n$ becomes larger and larger, the terms $\frac{1}{n}$ become smaller and smaller. As $n$ becomes infinitely large, $\frac{1}{n}$ become infinitely small. The sequence converges towards the limit $0$. Hence we write
\begin{equation}
  \label{eq:1}
  \lim_{n \to \infty} \frac{1}{n}  = 0
\end{equation}
As $n$ increases, eventually no computer, machine, or instrument will be able to distinguish the terms in this sequence from $0$, or from one another. This limit of the harmonic sequence will be regarded as a fundamental limit when we calculate the limits of rational formulas.

\noindent {\bf Example:}\\
Consider the sequence $b_n=\frac{n}{n+1}$. Some terms in this sequence are given in the table below, and are plotted in figure \vref{fig:inverse_harmonic_sequence}

\begin{center}
\begin{tabular}{|c|c|c|c|c|c|c|c|c|c|}
\hline
$b_1$ & $b_2$ & $b_3$ & $b_4$ & $b_5$ & $\cdots$ & $b_{99}$ & $\cdots$ & $b_{9999}$ & $\cdots$  \\
\hline
$\frac{1}{2}$ & $\frac{2}{3}$ & $\frac{3}{4}$ & $\frac{4}{5}$  & $\frac{5}{6}$  & $\cdots$ & 0.99 & $\cdots $  & 0.9999 & $\cdots $\\
\hline
\end{tabular}
\end{center}

\begin{figure}
  \centering
  \includegraphics[width=0.4\textwidth]{images/harmonic_sequence.eps}
  \caption{A plot of the terms in sequence $\frac{n}{n+1}$. The terms converge towards $1$. }
  \label{fig:inverse_harmonic_sequence}
\end{figure}
As $n \to \infty$, the numerator and denominator in $\frac{n}{n+1}$ become relatively equal in scale, and so the fraction approaches $1$. Hence we write
\begin{equation*}
  \lim_{n \to \infty} \frac{n}{n+1}  = 1
\end{equation*}
As $n$ increases, eventually no computer, machine, or instrument will be able to distinguish the terms in this sequence from $1$, or from one another.

Not all sequences have limits. A sequence can simply grow infinitely without bound, or can oscillate between values indefinitely.


\noindent {\bf Example:}\\
Let $c_n=n^2$. This sequence grows indefinitely, diverging towards $+ \infty$, and so no limit exists.
\begin{center}
\begin{tabular}{|c|c|c|c|c|c|c|c|}
\hline
$c_0$ & $c_1$ & $c_2$ & $c_3$ & $c_4$ & $c_5$ & $\cdots$ & $c_{1000}$  \\
\hline
$0$ & $1$ & $4$ & $9$ & $16$ & $25$ & $\cdots$ & $1000000$ \\
\hline
\end{tabular}
\end{center}

\noindent {\bf Example:}\\
Let $d_n=\begin{cases} +1, & n \text{ even} \\ -1, & n \text{ odd} \end{cases}$. This sequence oscillate alternately between $+1$ and $-1$. It never converges to any fixed value. Thus the sequence is said to be divergent.
\begin{center}
\begin{tabular}{|c|c|c|c|c|c||c|c|}
\hline
$d_0$ & $d_1$ & $d_2$ & $d_3$ & $d_4$ & $d_5$ & $\cdots$ & $d_{1000}$  \\
\hline
$0$ & $1$ & $4$ & $9$ & $16$ & $25$ & $\cdots$ & $1000000$ \\
\hline
\end{tabular}
\end{center}
\pagebreak
\subsection*{Limits of rational Formulas}

We will often need to calculate the limits of expressions like $\lim_{n\to \infty} \frac{3 n^2 + 5n +1}{2 n^2 +3}$. Numerical investigation reveals that this sequence is tending towards $\frac{3}{2}$, but we must be able to prove this. We cannot set $n$ equal to $\infty$ in the expression, as this would result in $\frac{\infty}{\infty}$ which is undefined/not a number.

Instead we will use the fundamental limit $\lim_{ n \to \infty} \frac{1}{n}=0$ to evaluate the limit of rational formulas. Because we know its limit at infinity is zero, we will adopt the convention that any $\left( \frac{1}{n}\right)$ terms in an expression can be set equal to zero when the limit at infinity is taken.

\noindent {\bf Example:}\\
\[ \lim_{n \to \infty} \left[ 7+\frac{3}{n} \right] =  \lim_{n \to \infty} \left[ 7+ 3 \left( \frac{1}{n} \right) \right] = 7+ 3(0) = 7\]

\noindent {\bf Example:}\\
\[ \lim_{n \to \infty} \left[ 8-\frac{4}{n}+\frac{5}{n^2} \right] =  \lim_{n \to \infty} \left[ 8-4 \left( \frac{1}{n} \right) + 5 \left( \frac{1}{n} \right) \left( \frac{1}{n} \right) \right] = 8-4(0)+5(0)(0) = 8\]

\noindent {\bf Example:}\\
\[
\lim_{n \to \infty } \frac{1}{n^3}= \lim_{n \to \infty } \left( \frac{1}{n} \right)\left( \frac{1}{n} \right)\left( \frac{1}{n} \right) = (0)(0)(0) = 0
\]

\noindent {\bf Example:}\\
\[
\lim_{n \to \infty } \frac{C}{n^p} = 0, \quad \text{if } p \ge 1, \quad \text{where $C$ is any constant}
\]

The limits of more complicated expressions involving $\frac{1}{n}$ are evaluated in the same way
\noindent {\bf Example:}\\
\[
\lim_{n \to \infty } \left[ \frac{5+\frac{2}{n}}{3-\frac{4}{n^2}} \right]= \frac{5+2(0)}{3-4(0)}=\frac{5}{3}
\]

\noindent {\bf Example:}\\
\[
\lim_{n \to \infty } \left[ \frac{21 -\frac{5}{n}+\frac{16}{n^3}}{4+\frac{9}{n^2}-\frac{10}{n^4}} \right]=\frac{21-0+0}{4+0-0}=\frac{21}{4}
\]

For general rational formulas, we need to transform the formula so that the fundamental limit $\left( \frac{1}{n} \right) \to 0$ can be applied. For example, consider the problem of finding the limit of $\frac{n}{n+1}$ as $n \to \infty $. To transform $\frac{n}{n+1}$, note that because
\[
\frac{\ \frac{1}{n}\ }{\ \frac{1}{n}\ }=1
\]
multiplying by this fraction will not change values. Using this we have
\[ \frac{n}{n+1} =   \frac{\ \frac{1}{n}\ }{\ \frac{1}{n}\ }\cdot \frac{n}{n+1} = \frac{\left( \frac{1}{n} \right) n}{\left( \frac{1}{n} \right)(n+1)} = \frac{\frac{n}{n}}{\frac{n}{n}+\frac{1}{n}}= \frac{1}{1+\frac{1}{n}}
\]
As the first and last formulas here are equal, the fundamental limit $\left( \frac{1}{n} \right) \to 0$ can now be applied as $n \to \infty $.

\[
\lim_{n \to \infty } \left[ \frac{n}{n+1} \right]= \lim_{n \to \infty } \left[ \frac{1}{1+\frac{1}{n}} \right]= \frac{1}{1+0}=1
\]
This result confirms the earlier observation that $\lim_{n \to \infty } \left[ \frac{n}{n+1} \right]=1$.

To find the limits of formulas containing higher powers of $n$, we need to divide the formula above and below by the highest power of $n$ in the expression.

\noindent {\bf Example:}\\
\noindent To find $\lim_{n \to \infty } \frac{n^2-7}{2n^2+n+3}$.

The highest power of $n$ in this expression is $n^2$. Therefore, we multiply the fraction above and below by $\frac{1}{n^2}$.
\begin{align*}
\frac{n^2-7}{2n^2+n+3}=\frac{\frac{1}{n^2}}{\frac{1}{n^2}}\cdot \frac{n^2-7}{2n^2+n+3} & = \frac{\left( \frac{1}{n^2} \right) \left( n^2-7 \right)}{ \left( \frac{1}{n^2} \right) \left( 2n^2+n+3 \right)}\\ & = \frac{\frac{n^2}{n^2}-\frac{7}{n^2}}{2 \cdot \frac{n^2}{n^2}+\frac{n}{n^2}+\frac{3}{n^2}} = \frac{1-\frac{7}{n^2}}{2+\frac{1}{n}+\frac{3}{n^2}}
\end{align*}
With these formulas equal, the fundamental limit $\left( \frac{1}{n} \right) \to 0$ can now be applied as $n \to \infty $.

\[
\lim_{n \to \infty } \left[ \frac{n^2-7}{2n^2+n+3} \right]=  \lim_{n \to \infty } \left[ \frac{1-\frac{7}{n^2}}{2+\frac{1}{n}+\frac{3}{n^2}}\right] = \frac{1-0}{2+0+0} = \frac{1}{2}
\]

\noindent {\bf Example:}\\
\noindent To find $\lim_{n \to \infty } \frac{2n-n^2}{5n^3+4n-10}$.

The highest power in the expression is $n^3$. Therefore, we multiply the fraction above and below by $\frac{1}{n^3}$. Before doing this however, note that the only place this term appears is in the denominator. Therefore, as $n \to \infty $, the denominator will grow much faster than other terms. In fact, the fraction will approach $\to \frac{-n^2}{5n^3}  = -\frac{1}{5n} \to 0$. But we will now confirm this formally.

\[
\frac{2n-n^2}{5n^3+4n-10} = \frac{\frac{1}{n^3}}{\frac{1}{n^3}}\cdot \frac{2n-n^2}{5n^3+4n-10} = \frac{2 \frac{n}{n^3}-\frac{n^2}{n^3}}{5 \frac{n^3}{n^3}+4 \frac{n}{n^3}-\frac{10}{n^3}} = \frac{\frac{2}{n^2}-\frac{1}{n}}{5+\frac{4}{n^2}-\frac{10}{n^3}}
\]
Hence
\[
\lim_{n \to \infty } \left[ \frac{2n-n^2}{5n^3+4n-10} \right]= \lim_{n \to \infty } \left[  \frac{\frac{2}{n^2}-\frac{1}{n}}{5+\frac{4}{n^2}-\frac{10}{n^3}} \right]= \frac{0-0}{5+0-0}=0
\]


\subsection*{Limits of Recursive Sequences}
If a recursive sequence has a limit $L$, this can sometimes be found by replacing all terms $a_{n}$ by $L$ in the recursive rule. Consider for example Heron's method to compute $\sqrt{S}$
\begin{align*}
  &\begin{cases}
    h_0=&G\\
    h_{n+1}=&\frac{1}{2} \left( h_n + \frac{S}{h_n} \right)
  \end{cases}
\end{align*}

First we assume that the sequence converges to some number $L$. Then $\lim_{n \to \infty} a_n=L$, and the terms in the sequence are become closer and closer to $L$ as $n$ increases. So for large $n$, $a_{n} \to  L$ and $a_{n+1}\to L$. Inserting these approximations into the recursion formula gives
\begin{align*}
  L&=\frac{1}{2}\left(L+S/L\right)\\
\Rightarrow L&=\frac{L}{2} + \frac{S}{2L}\\
\Rightarrow \frac{L}{2} &= \frac{S}{2L}, \quad \Rightarrow  L^2=S
\end{align*}
And so $L=\sqrt{S}$, as required.

However, it should be noted that this method finds the limit $L$ under the assumption that the limit exists; it will not work for sequences which have no limit. Care should be taken when applying this method to algorithms such as the Fast Reciprocal method, which can easily not have a limit if the initial guess $G$ is too poor.


\pagebreak
\subsection*{Limits in Relation to Computers}
\emph{The following section is a more advanced discussion of limits in relation to computers. This material will not be on the exam.}

The previous definition of the limit of a sequence is not entirely precise. A more precise definition of the limit of a sequence is the following.
\begin{framed}
A sequence $a_n$ is said to have a limit $L$ if:\\
For all $\epsilon > 0$ there is some integer $M>0$ such that for all $n>M$, $|a_n-L|< \epsilon$.
\end{framed}
The value $\epsilon$ can be considered to be the \emph{limit of precision}. This has relevance to computer science and how computers represent numbers.

Consider a computer which stores or represents numbers internally. Since storage space is finite, such a system can only represent numbers to a certain degree of accuracy. Infinite precision is not possible in practice. There is a limit to how precisely the computer(or any real measuring device) can represent numbers and moreover the differences between numbers.

In particular, on any computer, there will be a smallest non zero positive number $\epsilon$. While the computer can represent and store both $0$ and $\epsilon$ it cannot represent any number between them as these numbers are too small and beyond its limit of precision, and they will be equated to zero. Thus if $|a-b|<\epsilon$, the numbers $a$ and $b$ will be indistinguishable from one another on that computer.

Different computers and device have different limits of precision $\epsilon$. For double precision floating point numbers found on most computers $\epsilon \cong 5 \times 10^{-324}$—though numbers will become indistinguishable long before this due to rounding error $\cong  2 \times 10^{-16}$. In principle however, $\epsilon$ can be any non zero number, no matter how small.

But if a sequence has a limit, then for all $\epsilon > 0$ eventually $n$ will become large enough so that $|a_n -L|<\epsilon$, and so eventually no computer, no matter how precise, will be be able to distinguish between the terms $a_n$ and the limit $L$. The terms $a_n$ have come as close to $L$ as is possible on that computer; or put another way, we have estimated $L$ as accurately as we can on that device.

So a sequence with a limit $L$ offers a way to come as close as possible to $L$ on any computer. This is especially important for numbers like $\sqrt{2}$,$\sqrt{10}$ and $\pi$, which can never be represented exactly on any computer. Though these numbers cannot be represented exactly, if we have a sequence whose limit is such a number, then we can obtain as accurate an estimate of these numbers as we wish.

When you read stories of $\pi$ being approximated to the billionth or trillionth decimal place, sequences with $\lim_{n \to \infty} a_n = \pi$ are being used.

\end{document}
